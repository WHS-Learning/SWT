\chapter{Einführung}

\section{Projektbasierte vs. Produktbasierte Software}

\subsection{Projektbasierte Software}

Bei der Projektbasierten Software entwickelt der \textit{Auftragnehmer} für den \textit{Auftraggeber}. Der 
Auftraggeber hat die Geschäftsidee oder das Problem und erstellt die Vision der Software. Der Auftragnehmer 
verdient an der Software, nicht an der Vision. 

Der Auftraggeber kommuniziert also seine Vision an den Auftragnehmer, welcher dann sicherstellt, dass die Software
der Vision des Auftragsgebers entspricht.

\subsection{Produktbasierte Software}

Bei der Produktbasierten Software gibt es keine klare Trennung zwischen Auftraggeber und Auftragnehmer. Der 
Entwickler verdient an der Geschäftsidee. Es gibt hier keine formale Abnahme, sondern nur Akzeptanz am Markt. 
Die Software ist also nie richtig fertig.