\chapter{Übungsblatt 1}

\section{Aufgabe 1}

\subsection{Wiederholungsfragen aus ERP und ORP}

\textit{Was ist der Unterschied zwischen einer Klasse und einem Objekt?}

Ein Objekt ist eine Instanz einer Klasse. Die Klasse gibt den Bauplan vor und das Objekt existiert nach diesen.

\hrulefill 

\textit{Was ist der Unterschied zwischen public, private und protected?}

public ist für jede andere Klasse sichtbar, private nur innerhalb der Klasse und protected innerhalb der Klasse und allen Kindern der klasse.

\hrulefill

\textit{Was ist der Unterschied zwischen einem Interface und einer abstrakten Klasse?}

Eine abstrakte klasse gibt vor wie sich eine Kindklasse verhalten muss und ein Interface gibt ein versprechen, wie sich eine Klasse verhält.

\hrulefill 

\textit{Was ist der Unterschied zwischen statischen und nicht-statischen Methoden?}

Eine Statische methode kann nicht überschrieben werden. Außerdem ist sie in jeder Instanz der klasse identisch.

\hrulefill 

\textit{Was ist Kapselung? Finden Sie ein Beispiel im Code.}

\dots

\hrulefill

\textit{Was ist Polymorphie? Finden Sie ein Beispiel im Code.}

\dots

\hrulefill 

\textit{Was ist Vererbung? Finden Sie ein Beispiel im Code.}

\dots

\hrulefill

\textit{Finden Sie den Bug bei der Verwendung von Heiltränken?}

Wenn Energy hinzugefügt wird wodurch die Energy größer ist als die maxHealth wird der Aufruf einfach ins void geschickt und es passiert nix.

\subsection{Implementieren Sie (schriftlich, keine echte Implementierung):}

\textit{Was ist Kapselung? Finden Sie ein Beispiel im Code.}

\hrulefill

\textit{ein Gift, mit dem der Character 20 Lebenspunkte verliert.}

void poisen(Character char) {
    char.getHealth().addHealth(-20);
}

\hrulefill 

\textit{einen Heiltrank, mit dem der Character komplett geheilt wird.}

void healFull(Character char) {
    Health h = char.getHealth();
    h.addHealth(h.getMaxHealth() - h.getCurrentHealth());
}

\hrulefill 

\textit{einen Stärkungszauber, der die maximalen Lebenspunkte eines Characters um 30 erhöht.}

void increaseMaxHP(Character char) {
    Health h = char.getHealth();
    h.setMaxHealth(h.getMaxHealth() + 30)
}

\hrulefill

\subsection{Welche Teile des Codes mussten Sie ändern?}

Durch den Character Parameter egal. Wenn der Parameter nicht erwünscht ist muss das ganze in die Character klasse rein.

\section{Aufgabe 2}

\subsection{Beschreiben Sie die grundsätzlichen Unterschiede zwischen projektbasierter und produktbasierter Software.}

Bei der Projektbasierten Software handelt es sich um ein einmaldiges Unterfangen, welches nach Spezifikationen eines Auftragsgebers stattfindet. 
Produktbasiete Software wird dauerhaft erweitert und sofort an den Endkunden vertrieben.

\hrulefill

\subsection{Wie unterscheidet sich die Motivation eines Herstellers von projektbasierter Software von der eines Herstellers von produktbasierter Software?}

Bei Projektbasierter Software soll der Auftragsgeber zufriedengestellt werden und bei Produktbasierter Software verdient der Entwickler mit, die 
Software soll sich auf den Markt gut vertreiben.

\hrulefill

\subsection{Warum ist es oft wichtig, dass produktbasierte Software schnell auf den Markt gebracht wird?}

Bei Produktbasierter Software verdient der Entwickler erst Geld, sobald diese am Markt ist.

\hrulefill

\subsection{Warum ist es eine gute Idee, zuerst einen Prototypen entwickeln, bevor man ein neues Software-Produkt erstellt?}

Um zu gucken ob die Software gut am Markt ankommt und um früh Feedback zu bekommen.

\section{Aufgabe 3}

\subsection{Welche Herausforderungen und Probleme wird Software Engineering in den nächsten 20 Jahren zu bewältigen haben?}

\begin{itemize}
    \item KI
    \item Das 2035 Problem
    \item Depression haha (:
\end{itemize}